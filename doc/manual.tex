\documentclass{article}

%\usepackage[letterpaper, margin=1in]{geometry}
\usepackage[letterpaper]{geometry}
\usepackage{bold-extra}
\usepackage{bytefield}
\usepackage{tabularx}
\usepackage{threeparttable}
\usepackage{xifthen}
\usepackage{xparse}
\usepackage{xspace}

\newcommand{\systemname}[0]{68040pc\xspace}

\makeatletter
\newcommand*{\textoverline}[1]{$\overline{\hbox{#1}}\m@th$}
\makeatother

\newcommand{\textss}[1]{$_{\hbox{#1}}$}

\NewDocumentCommand{\addr}{mm}{%
  \IfNoValueTF{#2}
  	{\texttt{\$#1}}
	{\mbox{\texttt{\$#1}-\texttt{\$#2}}}
}

\newcommand{\memsection}[4]{
	\bytefieldsetup{bitheight=#3\baselineskip}
	\bitbox[]{10}{
		\texttt{#1}
		\\ \vspace{#3\baselineskip} \vspace{-2\baselineskip} \vspace{- #3pt}
		\texttt{#2}
	}
	\bitbox{16}{#4}
}

\begin{document}

\begin{titlepage}
	\begin{center}
		\vspace*{1cm}

		\Huge
		\textbf{\systemname Specification}

		\vspace{0.5cm}
		\LARGE
		A Personal Computer Platform

		\vspace{1.5cm}
		\textbf{Anthony Guerrero}

		\vfill

		\normalsize
		Document Revision 0.0.1

		\vspace{0.8cm}
	\end{center}
\end{titlepage}

\section{System Overview}

\subsection{Design Goal}
The primary objective of the \systemname is to construct a personal computer
platform that aligns with the capabilities and aesthetics of computing systems
from the 1990s. The system is not merely a reflection of retro hardware but a
modern interpretation emphasizing complete transparancy, user understanding,
and modifiability. The design principles prioritize open-source documentation,
source code, and hardware description language (HDL) to empower users and
enthusiasts to understand, modify, and expand upon the system.

\subsection{Key Features}
\begin{itemize}
	\item \textbf{Open Architecture:} An open-source ethos is at the core of the
	project. All documentation, designs, and code associated with this
	system will be freely accessible and modifiable by the end-users and
	community.

	\item \textbf{MC68040 Processor:} Serving as the heart of the system, this 32-bit
	microprocessor offers a blend of performance and functionality suitable
	for a wide range of computing tasks. It's Instruction Set Architecture
	(ISA) and built-in MMU lends itself well to implementing UNIX-like
	operating systems.

	\item \textbf{Flexible Expansion:} The platform is designed with expansion and
	modularity in mind, featuring an ``Expansion Bus'' architecture to
	facilitate the addition of new components. The bus dynamically assigns
	memory space to bus members, bypassing address conflicts and setup
	jumpers.

	\item \textbf{Integrated FPGA:} Utilizing the iCE40-HX4K FPGA, with full access
	to the CPU bus, most glue logic and peripheral controllers can be
	implemented and modified after the design is finalized.

	\item \textbf{Dynamic RAM Controller:} With four 72-pin SIMM slots, the platform
	supports up to 512MB of main system memory. (3.3v only!)

	\item \textbf{Standard Form Factor:} The PCB will be sized to fit in a pre-ATX
	form factor called ``Baby AT''. This allows for eight expansion slots,
	a DIN-5 sized cutout for a keyboard, and simple power management.
\end{itemize}

\subsection{System Philosophy}

Beyond the technical specifications, this project embodies a philosophy of
transparency, education, and user empowerment. The system is not just a
computing platform but a testament to the ethos that technology should be
understandable, modifiable, and, above all, open. Whether for education,
nostalgia, or innovation, this platform seeks to be a canvas for users to
explore the intricacies of computing, reminiscent of an era when personal
computing was just blossoming.

\section{Hardware}

\subsection{CPU}

The Motorola MC68LC040 was chosen because I found it in a box. The
\textoverline{MDIS} and \textoverline{IPLx} lines are tied high by pull-up
resistors during normal operation, which has the effect of enabling the small
output buffers and non-multiplexed bus mode on the 68040 CPU. Designing the
motherboard with this in mind makes the system also compatible with the standard
68040.

\subsection{SRAM}

There are 512KB of SRAM for the boot ROM to use. During boot, information about
the hardware installed in the system is collected into a device attribute table
and is communicated to the operating system.

\subsection{DRAM}

Four 72-pin SIMM slots provide the bulk RAM for the system. Each SIMM's
\textoverline{RAS} lines are tied together and sent to the memory controller,
\textoverline{CAS} lines are routed separately. The system should support
commodity 5V 72-pin SIMMs which were available for the 486 and early Pentium
platforms.

The SIMMs I have (Keystron MK16DS3232LP-50) have (16) 16M $\times$ 4-bit DRAMs
per module. The module is layed out as listed in Figure \ref{map:keystron}, the
DRAM chip numbers are listed from 0-7 for the front side and 8-15 for the back
side. I've ordered the table so Bank 0 is listed first in order of Data Line.
Next, Bank 1 is listed in the same order.

\begin{figure}[]
	\caption{Keystron MK16DS3232LP-50 module layout}
	\label{map:keystron}
\begin{center}
	\begin{tabularx}{0.75\textwidth}{X X X X}
		DRAM Chip \# & CAS Line & RAS Line & Data Lines\\
		\hline \\
		0  & \textoverline{CAS0} & \textoverline{RAS0} & D0-D3\\
		2  & \textoverline{CAS0} & \textoverline{RAS0} & D4-D7\\
		4  & \textoverline{CAS1} & \textoverline{RAS0} & D8-D11\\
		6  & \textoverline{CAS1} & \textoverline{RAS0} & D12-D15\\
		1  & \textoverline{CAS2} & \textoverline{RAS2} & D16-D19\\
		3  & \textoverline{CAS2} & \textoverline{RAS2} & D20-D23\\
		5  & \textoverline{CAS3} & \textoverline{RAS2} & D24-D27\\
		7  & \textoverline{CAS3} & \textoverline{RAS2} & D27-D31\\
		8  & \textoverline{CAS0} & \textoverline{RAS1} & D0-D3\\
		10 & \textoverline{CAS0} & \textoverline{RAS1} & D4-D7\\
		12 & \textoverline{CAS1} & \textoverline{RAS1} & D8-D11\\
		14 & \textoverline{CAS1} & \textoverline{RAS1} & D12-D15\\
		9  & \textoverline{CAS2} & \textoverline{RAS3} & D16-D19\\
		11 & \textoverline{CAS2} & \textoverline{RAS3} & D20-D23\\
		13 & \textoverline{CAS3} & \textoverline{RAS3} & D24-D27\\
		15 & \textoverline{CAS3} & \textoverline{RAS3} & D27-D31\\
	\end{tabularx}
\end{center}
\end{figure}

\subsection{Boot ROM}

The boot ROM program is stored on the same SPI EEPROM thats used to configure
the FPGA. Accesses to ROM space are routed through the FPGA, which performs the
required access to the SPI ROM and returns the data back to the CPU. This was
done to facilitate rapid prototyping by providing a fast and available interface
to update the ROM.

\subsection{I2C Bus}

I2C is used on-board for communication with the Real-Time Clock (DS1307) and
other devices. (Temperature sensors/Fan controllers, etc.)

\section{Memory Map}

The memory map of the system is mostly organized into 128MB sections, using the
\texttt{A[31:27]} signals to select them (Figure \ref{map:sysmem}). This means
the FPGA only needs 5 address lines to decode and provide chip select signals
for the on-board devices. It also takes \texttt{A[1:0]} from the MC68150 and
\texttt{A[18:2]} from the 68040 for addressing registers internal to the FPGA,
discriminating between devices in the System Registers space, and for bridging
up to 512KB of SPI flash to the system.

\begin{figure}[]
	\caption{System Memory Map}
	\label{map:sysmem}
	\vspace{2ex}
	\begin{bytefield}{12}
		\memsection{0000 0000}{}{2}{ROM Area}\\
		\memsection{0800 0000}{}{2}{On-board SRAM}\\
		\memsection{1000 0000}{}{2}{On-board DRAM}\\
		\memsection{1800 0000}{}{4}{-}\\
		\memsection{3800 0000}{}{2}{System Registers}\\
		\memsection{4000 0000}{}{8}{Expansion Area}\\
		\memsection{8000 0000}{}{4}{-}\\
		\memsection{B800 0000}{}{2}{EXP Configuration}\\
		\memsection{C000 0000}{FFFF FFFF}{4}{Expansion Area}\\
		% \memsection{ff00 0000}{ffff ffff}{3}{Expansion Card Area}\\
	\end{bytefield}
\end{figure}

\section{Expansion Bus}

The system provides a Zorro III-inspired expansion slot bus, the main feature of
which is dynamic memory mapping of add-in devices. 

\subsection{Design Goals}

\begin{itemize}
	\item \textbf{Expansion and Flexibility:} A primary aim of the expansion
	bus is to ensure that the design does not inherently limit future
	enhancements or adaptations.

	\item \textbf{Performance:} The bus shouldn't unnecessarily constrain
	the bandwidth of any peripheral that connects to it, including the
	bus-master. In practice, this means that it should pretty much be a
	minor extension to the local bus.

	\item \textbf{Support for Varied Devices:} The bus should accomodate any
	device which can interface with the MC68040's local bus. Unfortunately,
	this means 16 or 8-bit devices will need logic on the card to support
	``resizing'' the accesses.

	\item \textbf{Reliability:} Ensure robust operation and avoid
	complications that might make the system prone to errors.
\end{itemize}

\subsection{Bus Architecture}

The design of the expansion bus draws heavily from the Zorro III expansion bus
designed for the Amiga 3000 and 4000 series of computers, but differs in many
key ways which make it incompatible.

\subsubsection{Card Configuration}

Cards must ground the \textoverline{CDET} line to be included in the
bus, and therefore is a requirement for complying with this bus specification.
If the card doesn't pull this line down, or there isn't a card plugged into an
expansion slot, the \textoverline{CFGIN} line will be bridged to the
\textoverline{CFGOUT} line for that expansion slot.

\subsubsection{Bus Cycles}

\subsubsection{Arbitration/Mastering}

\subsubsection{Cache Support}

While the expansion bus doesn't have any cache consistency mechanisms for
managing caches between several caching bus masters, it does allow cards that
absolutely must not be cached to assert a cache inhibit line,
\textoverline{CI}, on a per-cycle basis. This cache management is mainly useful
for support of I/O and other devices that shouldn't be cached.

\subsubsection{Interrupts}

A card supporting interrupts has on-board registers to store one or more vector
numbers. The numbers are obtained from the OS by the device driver for the card
and the card/driver combination must be able to handle the situation in which no
additional vectors are available.

\subsection{Signal Description}

\subsubsection{Power Connections}

\begin{itemize}
	\item{\textbf{Digital Ground (GND)}} This is the digital supply ground
	used by all expansion cards as the return path for all expansion
	supplies.

	\item{\textbf{Main Supply (+5VDC)}} This is the main power supply for
	all expanion cards. (Define power specification)

\end{itemize}

\subsubsection{Clock Signals}

\begin{itemize}
	\item{\textbf{TBD...}}
\end{itemize}

\subsubsection{System Control Signals}

\begin{itemize}
	\item{\textbf{Hardware Bus Error}}

	\item{\textbf{System Interrupts}}
\end{itemize}

\subsubsection{Slot Control Signals}

\begin{itemize}

	\item{\textbf{Configuration Chain
	(\textoverline{\scshape CFGINn},
	\textoverline{\scshape CFGOUTn})}} The slot configuration mechanism uses
	the bus signals \textoverline{\scshape CFGOUTn} and
	\textoverline{\scshape CFGINn}, where ``\textsc{n}'' refers to the slot
	number. Each slot has its own version of both signals, which make up the
	\textit{configuration chain} between slots. Each subsequent
	\textoverline{CFGIN} is a result of all previous
	\textoverline{CFGOUT}s, going from slot 0 to the last slot on the
	expansion bus.

	During the autoconfiguration process, an unconfigured card responds to
	the address space X if its \textoverline{\scshape CFGINn} is asserted.
	All unconfigured cards start up with \textoverline{\scshape CFGOUTn}
	negated. When configured, a card will assert its
	\textoverline{\scshape CFGOUTn} which results in the
	\textoverline{\scshape CFGINn} of the next slot being asserted.
	Backplane logic automatically passes on the state of the previous
	\textoverline{\scshape CFGOUTn} to the next
	\textoverline{\scshape CFGINn} for any slot not occupied by a card.

	\item{\textbf{Card Detect (\textoverline{CDET})}} This signal is to
	always be attached to ground by cards to allow them to participate in
	the expansion bus. This signal is part of the backplane circuitry that
	allows the configuration chain to function.

\end{itemize}

\subsubsection{DMA Control Signals}

\begin{itemize}
	\item{\textbf{TBD...}}
\end{itemize}

\subsubsection{Address and Related Control Signals}

\begin{itemize}

	\item{\textbf{Address Bus (A\textss{0}-A\textss{31})}} This is the
	expansion address bus, which is driven by the bus master.

\end{itemize}

\subsubsection{Data and Related Control Signals}

\begin{itemize}
	\item{\textbf{Data Bus (D\textss{0}-D\textss{31})}} This is the
	expansion data bus, which is driven by either the master or the slave
	when ``'' is asserted by the master. It's valid for reads when
	\textoverline{DTACK} is asserted by the slave.

	\item{\textbf{Transfer Acknowledge (\textoverline{TA})}} This
	signal is used to normally terminate an expansion bus cycle. The slave
	is always responsible for driving this signal. For a read cycle, it
	asserts \textoverline{TA} as soon as it has driven valid data onto
	the bus. For a write cycle, it asserts \textoverline{TA} as soon as
	it's done with the data.

\item{\textbf{Transfer Burst Inhibit/Transfer Cache Inhibit (\textoverline{TBI}/\textoverline{TCI})}}
	This line is asserted at the same time as \textoverline{TA} to indicate
	to the bus master that the cycle must not be cached, or that that the
	cycle can't be completed as a burst transfer.
\end{itemize}

\subsection{Electrical Specifications}

\subsubsection{Standard Signals}

The majority of signals on the bus are in this group. These are the bussed
signals, driven actively on the bus by F-series (or compatible) drivers, usually
tri-stated when ownership of the signal changed for master and slave, and
generally terminated with a 220$\Omega$/330$\Omega$ thevenin terminator.

\begin{center}
	\begin{tabularx}{0.75\textwidth}{X X X}
		A\textss{0}-A\textss{31} & D\textss{0}-D\textss{31} & R/\textoverline{W} \\
		FC\textss{0}-FC\textss{2} & & 
	\end{tabularx}
\end{center}

\subsubsection{Clock Signals}

\subsubsection{Open Collector Signals}

Many of the bus signals are shared via open collector or open drain outputs
rather than via tri-stated signals. A backplane resistor pulls these lines high,
cards only drive the line low.

\begin{center}
	\begin{tabularx}{0.75\textwidth}{X X X}
		\textoverline{TA} & \textoverline{TCI}/\textoverline{TBI} & \textoverline{TEA} \\
		\textoverline{RESET} & R/\textoverline{W} & 
	\end{tabularx}
\end{center}

\subsubsection{Non-bussed signals}

\begin{center}
	\begin{tabularx}{0.75\textwidth}{X X X}
		\textoverline{\scshape CFGINn} & \textoverline{\scshape IRQn} & R/\textoverline{W} \\
		\textoverline{\scshape CFGOUTn} & & 
	\end{tabularx}
\end{center}

\subsubsection{Slot Power Availability}

The system power for the expansion bus is based on the slot configurations. A
backplane is always free to supply extra power, but it must meet the minimum
requirements specified here. All cards must be designed with the minimum
specifications in mind, especially the tolerances.

\begin{center}
	\begin{tabularx}{0.75\textwidth}{X X}
		Pin & Supply \\
		\hline \\
		X, X	& $+$5 VDC $\pm$5\% @ 2A \\
		X	& $+$12 VDC $\pm$5\% @ 500mA \\
	\end{tabularx}
\end{center}



\subsection{Mechanical Specifications}

\subsection{Expansion Slot Pin Assignments}

%\begin{figure}[h]
	\begin{centering}

	\begin{threeparttable}
	\caption{Expansion Bus Connector Pinout}
	\begin{tabularx}{\textwidth}
		{| c | X || c | X || c | X |}
		\hline
		Pin & Name & Pin & Name & Pin & Name \\
		\hline\hline
		a1  & NC\tnote{1}	& b1  &	D\textss{16}	& c1  &	NC 		\\
		\hline
		a2  & NC		& b2  &	D\textss{17}	& c2  &	NC 		\\
		\hline
		a3  & D\textss{0}	& b3  &	D\textss{18}	& c3  &	NC 		\\
		\hline
		a4  & D\textss{2}	& b4  &	D\textss{19}	& c4  &	NC 		\\
		\hline
		a5  & D\textss{4}	& b5  &	D\textss{20}	& c5  &	NC 		\\
		\hline
		a6  & D\textss{6}	& b6  &	D\textss{21}	& c6  &	NC 		\\
		\hline
		a7  & D\textss{8}	& b7  &	D\textss{22}	& c7  &	NC 		\\
		\hline
		a8  & D\textss{10}	& b8  &	D\textss{23}	& c8  &	NC 		\\
		\hline
		a9  & D\textss{12}	& b9  &	D\textss{24}	& c9  &	NC 		\\
		\hline
		a10 & D\textss{14}	& b10 &	D\textss{25}	& c10 &	NC 		\\
		\hline
		a11 & GND		& b11 &	D\textss{26}	& c11 &	NC 		\\
		\hline
		a12 & A\textss{1}	& b12 &	D\textss{27}	& c12 &	NC 		\\
		\hline
		a13 & A\textss{3}	& b13 &	D\textss{28}	& c13 &	NC 		\\
		\hline
		a14 & A\textss{5}	& b14 &	D\textss{29}	& c14 &	NC 		\\
		\hline
		a15 & A\textss{7}	& b15 &	D\textss{30}	& c15 &	NC 		\\
		\hline
		a16 & A\textss{9}	& b16 &	D\textss{31}	& c16 &	NC 		\\
		\hline
		a17 & A\textss{11}	& b17 &	A\textss{0}	& c17 &	NC 		\\
		\hline
		a18 & A\textss{13}	& b18 &	A\textss{0}	& c18 &	NC 		\\
		\hline
		a19 & A\textss{15}	& b19 &	NC		& c19 &	NC 		\\
		\hline
		a20 & A\textss{17}	& b20 &	NC		& c20 &	NC 		\\
		\hline
		a21 & A\textss{19}	& b21 &	NC		& c21 &	NC 		\\
		\hline
		a22 & NC		& b22 &	NC		& c22 &	NC 		\\
		\hline
		a23 & NC		& b23 &	NC		& c23 &	NC 		\\
		\hline
		a24 & NC		& b24 &	NC		& c24 &	NC 		\\
		\hline
		a25 & NC		& b25 &	NC		& c25 &	NC 		\\
		\hline
		a26 & NC		& b26 &	NC		& c26 &	NC 		\\
		\hline
		a27 & NC		& b27 &	NC		& c27 &	NC 		\\
		\hline
		a28 & NC		& b28 &	NC		& c28 &	NC 		\\
		\hline
		a29 & NC		& b29 &	NC		& c29 &	NC 		\\
		\hline
		a30 & NC		& b30 &	NC		& c30 &	NC 		\\
		\hline
		a31 & NC		& b31 &	NC		& c31 &	NC 		\\
		\hline
		a32 & NC		& b32 &	NC		& c32 &	NC 		\\
		\hline
	\end{tabularx}
	\begin{tablenotes}
	\item [1] This means it's not connected. You knew that though. This
	footnote is only here so I remember how to do it later.
	\end{tablenotes}
	\end{threeparttable}
	\end{centering}
% \end{figure}

\section{Interrupt System}

\section{Special Thanks}
\setlength\parindent{0pt}

\textbf{Ben Eater} for creating engaging videos on computer engineering,
inspiring me to build a computer of my own.

\textbf{Lawrence Manning} MAXI030

\textbf{Stephen Moody} Y Ddraig(030)

\textbf{Dave Haynie} Author of ``The Zorro III Bus Specification''

\end{document}
